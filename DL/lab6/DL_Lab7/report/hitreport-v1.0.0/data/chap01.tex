% !TeX root = ../hitreport-example.tex

\chapter{实验介绍}

\section{选题说明}

随着计算机科学和图像处理技术的发展,图像在农业、工业、医学、科研等领域中被广泛使用。然而,图像在成像和传输过程中产生的噪声对图片本身的质量产生了很大的干扰,这可能导致图像的利用价值大打折扣。因此,图像去噪问题成为了底层计算机视觉领域的一个经典问题,旨在设法对图像中的噪声进行抑制和衰减,使得有用的信息得到加强,便会在很大程度上有利于我们对于图像信息的辨别、理解和应用。

假设一张没有噪声的图像(下称原图)$ \boldsymbol{x} $,产生了噪声后变为$ \boldsymbol{y} = \boldsymbol{x} + \boldsymbol{n} $。其中,$ \boldsymbol{n} $是附加在图像上的噪声,这类噪声被称为加性噪声。我们一般假设$ \boldsymbol{n_{ij}} \sim N(0, \sigma^2) $,即作用在每个像素上的噪声值服从均值为0,方差为$ \sigma^2 $的高斯分布,这样的噪声也被称作加性高斯白噪声(AWGN)。其中$ \sigma $在图像去噪问题中又被称作噪声等级,是可以人为设定的参数。图像去噪的目的,就是给定一张含有噪声的图像$ \boldsymbol{y} $,通过机器学习等方法得到尽可能接近原图$ \boldsymbol{x} $的图像$ \boldsymbol{\hat{x}} $。

DnCNN网络是一个十分经典的基于卷积神经网络结构的去噪模型。该模型使用残差学习(Resitual Learning)技术配合批标准化(Batch Normalization)技巧,达到了很好的去噪性能。并且,由于该模型是基于卷积神经网络实现的,因此可以充分利用GPU硬件进行加速,这使得该模型在时间性能上也极为可观,对单张图像的去噪时间可以达到亚毫秒级别。而本次实验的内容,就是复现DnCNN去噪模型,并使用其对图像进行去噪,并对去噪性能和时间性能进行测试和讨论。

\section{任务描述}

本次实验的任务大致分为以下几条:
\begin{itemize}
	\item 阅读DnCNN模型的相关文献,了解DnCNN模型的网络架构和设计的技巧点,掌握模型去噪的基本原理;
	\item 基于PyTorch和CUDA复现模型的代码,包括训练和测试代码;
	\item 使用训练代码和指定的训练集对模型进行训练,并生成训练的log文本;
	\item 使用测试代码和指定的测试集对训练好的模型进行测试;
	\item 对测试结果进行讨论。
\end{itemize}

\section{数据集描述}

\subsection{数据集的选取}
在本次实验中,模型的训练集采用了BSD400~\cite{chen2016trainable}数据集。该数据集中包含了400张180$\times$180大小的灰度图,其中部分图片如图~\ref{fig:bsd400sample}所示。测试集选用了BSD68~\cite{roth2005fields}和Set12~\cite{zhang2017beyond}数据集。BSD68由来自BSD数据集测试集的68张图片组成,大小为321$\times$481,其中部分图片如图~\ref{fig:bsd68sample}所示。Set12由12张图片组成,其中7张图片大小为256$\times$256,5张图片大小为512$\times$512,其中部分图片如图~\ref{fig:set12sample}所示。

\begin{figure}
	\centering
	\subcaptionbox*{}
	{\includegraphics[width=0.09\linewidth]{figures/train_1}}
	\subcaptionbox*{}
	{\includegraphics[width=0.09\linewidth]{figures/train_2}}
	\subcaptionbox*{}
	{\includegraphics[width=0.09\linewidth]{figures/train_3}}
	\subcaptionbox*{}
	{\includegraphics[width=0.09\linewidth]{figures/train_4}}
	\subcaptionbox*{}
	{\includegraphics[width=0.09\linewidth]{figures/train_5}}
	\subcaptionbox*{}
	{\includegraphics[width=0.09\linewidth]{figures/train_6}}
	\subcaptionbox*{}
	{\includegraphics[width=0.09\linewidth]{figures/train_7}}
	\subcaptionbox*{}
	{\includegraphics[width=0.09\linewidth]{figures/train_8}}
	\subcaptionbox*{}
	{\includegraphics[width=0.09\linewidth]{figures/train_9}}
	\subcaptionbox*{}
	{\includegraphics[width=0.09\linewidth]{figures/train_10}}
	\caption{BSD400中部分照片展示}
	\label{fig:bsd400sample}
\end{figure}

\begin{figure}
	\centering
	\subcaptionbox*{}
	{\includegraphics[width=0.09\linewidth]{figures/test_bsd68_01}}
	\subcaptionbox*{}
	{\includegraphics[width=0.09\linewidth]{figures/test_bsd68_02}}
	\subcaptionbox*{}
	{\includegraphics[width=0.09\linewidth]{figures/test_bsd68_03}}
	\subcaptionbox*{}
	{\includegraphics[width=0.09\linewidth]{figures/test_bsd68_04}}
	\subcaptionbox*{}
	{\includegraphics[width=0.09\linewidth]{figures/test_bsd68_05}}
	\subcaptionbox*{}
	{\includegraphics[width=0.09\linewidth]{figures/test_bsd68_06}}
	\subcaptionbox*{}
	{\includegraphics[width=0.09\linewidth]{figures/test_bsd68_07}}
	\subcaptionbox*{}
	{\includegraphics[width=0.09\linewidth]{figures/test_bsd68_08}}
	\subcaptionbox*{}
	{\includegraphics[width=0.09\linewidth]{figures/test_bsd68_09}}
	\subcaptionbox*{}
	{\includegraphics[width=0.09\linewidth]{figures/test_bsd68_10}}
	\caption{BSD68中部分照片展示}
	\label{fig:bsd68sample}
\end{figure}

\begin{figure}
	\centering
	\subcaptionbox*{}
	{\includegraphics[width=0.09\linewidth]{figures/test_set12_01}}
	\subcaptionbox*{}
	{\includegraphics[width=0.09\linewidth]{figures/test_set12_02}}
	\subcaptionbox*{}
	{\includegraphics[width=0.09\linewidth]{figures/test_set12_03}}
	\subcaptionbox*{}
	{\includegraphics[width=0.09\linewidth]{figures/test_set12_04}}
	\subcaptionbox*{}
	{\includegraphics[width=0.09\linewidth]{figures/test_set12_05}}
	\subcaptionbox*{}
	{\includegraphics[width=0.09\linewidth]{figures/test_set12_06}}
	\subcaptionbox*{}
	{\includegraphics[width=0.09\linewidth]{figures/test_set12_07}}
	\subcaptionbox*{}
	{\includegraphics[width=0.09\linewidth]{figures/test_set12_08}}
	\subcaptionbox*{}
	{\includegraphics[width=0.09\linewidth]{figures/test_set12_09}}
	\subcaptionbox*{}
	{\includegraphics[width=0.09\linewidth]{figures/test_set12_10}}
	\caption{Set12中部分照片展示}
	\label{fig:set12sample}
\end{figure}

\subsection{数据的预处理}
在训练和测试之前,均需要对图片进行归一化处理,具体公式见式~\ref{for:norm}。其中$ x_{ij} $表示图像中第$ i $行第$ j $列位置处的像素值。255为8位像素值所能表示的最大值,也就是说,该归一化操作将图片的像素值全部转换到[0, 1]区间内。
\begin{align}\label{for:norm}
	x_{ij}^{\prime} = \frac{x_{ij}}{255.0}
\end{align}

此外,在对模型进行训练时,我们并不是使用整张图像对模型进行训练,而是首先将图像分成预定大小的patch,然后为每个patch加上高斯噪声,形成训练时的输入,而原来没有噪声的patch则是模型的监督数据。根据论文中提出的训练技巧,使用多尺度融合机制和数据增广可以进一步提升模型的训练效果。因此,在对训练数据进行预处理时,还需要将图像进行多种比例的缩放以及多种方式的增广,然后再按照规定的patch size裁剪成若干的patch,作为模型的训练数据。这些预处理工作的具体代码实现见图~\ref{fig:preprocess}。这段预处理程序依次对图像进行1、0.9、0.8、0.7比例的缩放,裁剪出patch后随机采用翻转、旋转、先旋转后翻转等若干增广方式中的一种对该patch进行增广处理。预处理结束后,将所有生成的patch作为模型的训练集数据。

\begin{figure}
	\centering
	\subcaptionbox*{}
	{\includegraphics[width=1\linewidth]{figures/preprocess1}}
	\subcaptionbox*{}
	{\includegraphics[width=1\linewidth]{figures/preprocess2}}
	\caption{部分数据预处理代码实现}
	\label{fig:preprocess}
\end{figure}


