% !TeX root = ../hitreport-example.tex

\chapter{实验结果分析}\label{chapter:res}
\section{测试结果}
BSD68数据集上的部分测试结果如图~\ref{fig:bsd68res}所示。

\begin{figure}
	\centering
	\subcaptionbox*{}
	{\includegraphics[width=0.45\linewidth]{figures/res_bsd68_1}}
	\subcaptionbox*{}
	{\includegraphics[width=0.45\linewidth]{figures/res_bsd68_2}}
	
	\subcaptionbox*{}
	{\includegraphics[width=0.45\linewidth]{figures/res_bsd68_3}}
	\subcaptionbox*{}
	{\includegraphics[width=0.45\linewidth]{figures/res_bsd68_4}}
	\caption{BSD68中部分测试结果展示}
	\label{fig:bsd68res}
\end{figure}

Set12数据集上的部分测试结果如图~\ref{fig:set12res}所示。
\begin{figure}
	\centering
	\subcaptionbox*{}
	{\includegraphics[width=0.45\linewidth]{figures/res_set12_1}}
	\subcaptionbox*{}
	{\includegraphics[width=0.45\linewidth]{figures/res_set12_2}}
	
	\subcaptionbox*{}
	{\includegraphics[width=0.45\linewidth]{figures/res_set12_3}}
	\subcaptionbox*{}
	{\includegraphics[width=0.45\linewidth]{figures/res_set12_4}}
	\caption{Set12中部分测试结果展示}
	\label{fig:set12res}
\end{figure}


两个数据集上的峰值信噪比(PSNR)和结构相似性(SSIM)如表~\ref{table:res}所示。

\begin{table}[]
	\centering
	\caption{测试结果}
	\begin{tabular}{ccc}
		\hline
		& PSNR (dB) & SSIM \\ \hline
		BSD68 & 29.22 & 0.9011 \\
		Set12 & 30.43 & 0.9295 \\ \hline
	\end{tabular}
	\label{table:res}
\end{table}


\section{方案评价}
从图~\ref{fig:bsd68res}和图~\ref{fig:set12res}中可以看出,无论是大尺寸的照片,还是小尺寸的照片,都能够取得良好的效果,噪声被有效地去除,能够还原图片的真实质量。

从表~\ref{table:res}中可以看出,两个数据集上的峰值信噪比只有30,结构相似性能够达到0.9和0.93左右,在数值上证明了方法的有效性。