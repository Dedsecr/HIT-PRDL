% !TeX root = ../hitreport-example.tex

% 中英文摘要和关键字

\begin{abstract}
	针对图像去噪问题,目前卷积神经网络是最多被使用的模型之一。许多模型只利用了卷积神经网络感受野内全部像素组成的局部信息进行去噪,而没有使用图像中潜在的一些非局部信息。为了探究这些潜在的信息是否能够改善图像去噪的效果,本文在现有去噪模型的基础之上,将每个图像块在噪声图中的相似图像块作为一个非局部信息加入到去噪的过程中。最终通过对模型性能的测试,本文证实了在一些具有规则图案和重复结构的图像上,引入上述的非局部特征信息对去噪的效果有一定的提升作用。不过,非局部信息的提取也是一个比较耗时的过程,这也增加了相应的时间成本。

  % 关键词用“英文逗号”分隔,输出时会自动处理为正确的分隔符
  \hitsetup{
    keywords = {图像去噪, 非局部信息, 卷积神经网络},
  }
\end{abstract}

