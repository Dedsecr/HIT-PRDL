% !TeX root = ./hitreport-example.tex

% 基本信息配置

\hitsetup{
  %******************************
  % 注意:
  %   1. 配置里面不要出现空行
  %   2. 不需要的配置信息可以删除
  %   3. 建议先阅读文档中所有关于选项的说明
  %******************************
  %
  % 校区选择
  %   可供选择为harbin, shenzhen, weihai,必须填写
  campus = {harbin},
  %
  % 标题
  %   可使用“\\”命令手动控制换行
  %
  title  = {模式识别与深度学习实验六},
  %
  % 副标题
  %   可使用“\\”命令手动控制换行
  %
  expand = {DnCNN去噪模型复现及使用},
  %
  % 所在学院(部)
  %   填写所属学院(部)的全名
  %
  department = {计算机科学与技术学院},
  %
  % 专业
  %   填写专业名称
  %
  discipline  = {智能信息处理},
  %
  % 姓名
  %
  classnum  = {1903103},
  %
  % 班级
  %
  author  = {王雨桐 石翔宇},
  %
  % 学号
  %
  student-id = {1190200527},
  %
  % 指导教师
  %
  supervisor  = {左旺孟},
  %
  % 实验地点
  %   暂未区分课程报告封面与实验报告封面,课程报告可随意填写
  %
  lablocation = {},
  %
  % 报告学期
  %
  term = {2022春季学期},
  %
  % 日期
  %   使用 ISO 格式;默认为当前时间
  %
  date = {2022-06-08},
  %
}

% 载入所需的宏包

% 可以使用 nomencl 生成符号和缩略语说明
% \usepackage{nomencl}
% \makenomenclature

% 表格加脚注
\usepackage{threeparttable}

% 表格中支持跨行
\usepackage{multirow}

% 固定宽度的表格。
% \usepackage{tabularx}

% 跨页表格
\usepackage{longtable}

% 量和单位
\usepackage{siunitx}

% 定理类环境宏包
\usepackage{amsthm}
% 也可以使用 ntheorem
% \usepackage[amsmath,thmmarks,hyperref]{ntheorem}

% 参考文献使用 BibTeX + natbib 宏包
% 顺序编码制
%\usepackage[sort]{natbib}
%\bibliographystyle{thuthesis-numeric}

% 著者-出版年制
% \usepackage{natbib}

% 声明 BibLaTeX 的数据库
% \addbibresource{ref/refs.bib}

% 定义所有的图片文件在 figures 子目录下
\graphicspath{{figures/}}

% 数学命令
\newcommand\dif{\mathop{}\!\mathrm{d}}  % 微分符号

% hyperref 宏包在最后调用
\usepackage{hyperref}

